\section{Algorithms}

\subsection{number theory}
\subsubsection{linear sieve}
\begin{lstlisting}
/*
 *
 * Eratosthenes' sieve.
 *
 * Complexity: O(n).
 *
 * Usage:
 *   - lp[x] = minimal y such that x divisible by y
 *   - pr contains all primes <= N
 *
 */

const int N = (int)1e7;
int lp[N + 1];
vector<int> pr;
 
void sieve() {
    for (int i = 2; i <= N; ++i) {
        if (lp[i] == 0) {
            lp[i] = i;
            pr.push_back(i);
        }
        for (int j = 0; j < (int)pr.size() &&
                pr[j] <= lp[i] && i * pr[j] <= N; ++j)
            lp[i * pr[j]] = pr[j];
    }
}
\end{lstlisting}
\subsubsection{fft}
\begin{lstlisting}
/*
 *
 * Fast Fourier Transformation.
 *
 * Complexity: O(NlogN).
 *
 * You can calc fft for 2 polynomials in one run:
 *   fa = A(x) + i*B(x)
 *   fa = fft(fa)
 *
 *   fft(A) = (fa[i] + conj(fa[(n - i) % n])) / base(2, 0)
 *   fft(B) = (fa[i] - conj(fa[(n - i) % n])) / base(0, 2)
 *
 * And you can calc inverse fft:
 *   fb = fft(A) + i*fft(B)
 *   fb = inverse_fft(fb)
 *
 *   A = fb[i].real()
 *   B = fb[i].imag()
 */

const double PI = acos(-1.0);

const int MAXN = 550500;

typedef complex<double> base;

int rev[MAXN];
base fa[MAXN];


void fft(base *a, int lg, bool inv) {
    int n = (1 << lg);
    for(int i = 1; i < n; i++) {
        rev[i] = (rev[i >> 1] >> 1) | ((i & 1) << (lg - 1));
        if(rev[i] < i)
            swap(a[i], a[rev[i]]);
    }

    for(int len = 2; len <= n; len <<= 1) {
        double ang = 2 * PI / len;
        if (inv)
            ang *= -1.0;
        base w(1, 0), wn(cos(ang), sin(ang));
        for(int j = 0; j < (len >> 1); j++, w = w * wn)
            for(int i = 0; i < n; i += len) {
                base u = a[i + j], v = w * a[i + j + (len >> 1)];
                a[i + j] = u + v;
                a[i + j + (len >> 1)] = u - v;
            }
    }
    if (inv) {
        for(int i = 0; i < n; i++)
            a[i] /= n;
    }
}


void multiply(const vector<int> &a, const vector<int> &b, vector<int> &res) {
    int n = 1, lg = 1;
    while (n < max (a.size(), b.size()))
        n <<= 1, ++lg;
    n <<= 1;
    for (int i = 0; i < n; ++i) {
        if (i < a.size() && i < b.size())
            fa[i] = base(a[i], b[i]);
        else if (i < a.size())
            fa[i] = base(a[i], 0);
        else if (i < b.size())
            fa[i] = base(0, b[i]);
        else
            fa[i] = base(0, 0);
    }

    fft (fa, lg, false);
    for (int i = 0; i < n; ++i)
        fa[i] *= fa[i];
    fft (fa, lg, true);

    res.resize(n);
    for (int i = 0; i < n; ++i)
        res[i] = (int)(fa[i].imag() / 2.0 + 0.5);
}

\end{lstlisting}
\subsection{optimizations}
\subsubsection{fast io}
\begin{lstlisting}
/*
 * Increases speed of reading and writting integers.
 *
 */

void writeInt(int x) {
    char s[10];
    int n = 0;
    while (x || !n)
        s[n++] = '0' + x % 10, x /= 10;
    while (n--)
        putc(s[n], stdout);
    putc('\n', stdout);
}

int readInt() {
    int s = 1, x = 0, c = getc(stdin);
    while (c <= 32)
        c = getc(stdin);
    if (c == '-')
        s = -1, c = getc(stdin);
    while ('0' <= c && c <= '9')
        x = x * 10 + c - '0', c = getc(stdin);
    return x * s;
}
\end{lstlisting}
\subsubsection{operator new}
\begin{lstlisting}
/*
 * Increases speed of memory allocating and deleting.
 * Memory doesn't reallocates. May cause MLE or RE.
 */

const int SZ = 5e7;
char buf[SZ];
int pos;

void * operator new(size_t x) {
    pos += x;
    if (pos > SZ) {
        throw 42;
    }
    return buf + pos - x;
}

void operator delete(void *x) {}
\end{lstlisting}
\subsection{geometry}
\subsubsection{convex hull}
\begin{lstlisting}
/*
 * Convex hull(Graham's traversal).
 *
 * Complexity: O(NlogN)
 *
 * Requrements: "point class"
 */

bool cmp(Point a, Point b) {
     return a.x < b.x || a.x == b.x && a.y < b.y;
}

bool cw(Point a, Point b, Point c) {
     return a.x * (b.y - c.y) +
            b.x * (c.y - a.y) +
            c.x * (a.y - b.y) < 0;
}

bool ccw(Point a, Point b, Point c) {
     return a.x * (b.y - c.y) +
            b.x * (c.y - a.y) +
            c.x * (a.y - b.y) > 0;
}

void convex_hull(vector<Point>& a) {
     if (a.size() == 1) return;
     sort(a.begin(), a.end(), &cmp);
     Point p1 = a[0];
     Point p2 = a.back();
     vector<Point> up, down;
     up.push_back(p1);
     down.push_back(p1);
     for (size_t i = 1; i < a.size(); ++i) {
          if (i == a.size() - 1 || cw(p1, a[i], p2)) {
               while (up.size() >= 2 &&
                      !cw(up[up.size() - 2], up[up.size() - 1], a[i])) {
                    up.pop_back();
               }
               up.push_back (a[i]);
          }
          if (i == a.size() - 1 || ccw(p1, a[i], p2)) {
               while (down.size() >= 2 &&
                      !ccw(down[down.size() - 2], down[down.size() - 1], a[i])) {
                    down.pop_back();
               }
               down.push_back(a[i]);
          }
     }
     a.clear();
     for (size_t i = 0; i < up.size(); ++i) {
          a.push_back(up[i]);
     }
     for (size_t i = down.size() - 2; i > 0; --i) {
          a.push_back(down[i]);
     }
}

\end{lstlisting}
\subsubsection{point}
\begin{lstlisting}
/*
 * This file contains realization of Point class.
 *
 * Requirements: "iostream", "cmath", "double_utils"
 */

class Point {
 public:
    double x;
    double y;
    Point() = default;
    Point(double x, double y) : x(x), y(y) {}

    void scan() {
        scanf("%lf %lf", &x, &y);
    }

    void print() const {
        printf("%.15lf %.15lf\n", x, y);
    }

    Point operator+(const Point& other) const {
        return Point(x + other.x, y + other.y);
    }

    Point operator-(const Point& other) const {
        return Point(x - other.x, y - other.y);
    }

    Point operator-() const {
        return Point(-x, -y);
    }

    Point operator*(double k) const {
        return Point(x * k, y * k);
    }

    Point operator/(double k) const {
        return Point(x / k, y / k);
    }

    bool operator==(const Point& other) const {
        return double_equal(x, other.x) && double_equal(y, other.y);
    }

    double operator%(const Point& other) const {
        return x * other.x + y * other.y;
    }

    double operator*(const Point& other) const {
        return x * other.y - y * other.x;
    }

    double Length() const {
        return sqrt(*this % (*this));
    }

    double DistTo(const Point& other) const {
        return (*this - other).Length();
    }

    double DistTo(const Point& first, const Point& second) const {
        double dist = first.DistTo(second);
        if (double_equal(dist, 0)) {
            return -1;
        }
        double area = (*this - first) * (*this - second);
        return fabs(area) / dist;
    }

    Point Normalize(double k = 1) const {
        double len = Length();
        if (double_equal(len, 0) ) {
            return Point();
        }
        return *this * (k / len);
    }

    Point GetH(const Point & A, const Point & B) const {
        Point C = *this;
        Point v = B - A;
        Point u = C - A;
        double k = v % u / v.Length();
        v = v.Normalize(k);
        Point H = A + v;
        return H;
    }
};
\end{lstlisting}
\subsubsection{planes check}
\begin{lstlisting}
/*
 * Checks if union of half-planes has non-zero area.
 *
 * Expected complexity: O(n).
 *
 */


#define ld long double

const ld inf = 1e9;
const ld eps = 1e-9;

inline ld det(ld a, ld b, ld c, ld d) {
    return a * d - b * c;
}

struct line {
    ld a, b, c;
    line(ld a = 0, ld b = 0, ld c = 0) : a(a), b(b), c(c) {}
    line(pt p, pt q) {
        a = p.y - q.y;
        b = q.x - p.x;
        c = -a * p.x - b * p.y;
        ld g = hypot(a, b);
        a /= g;
        b /= g;
        c /= g;
    }

    ld get_y(ld x) const {
        return -(a/b * x + c / b);
    }
    ld get_val(pt p) const {
        return a * p.x + b * p.y + c;
    }

    pt intersect(const line& p) const {
        ld d = det(a, b, p.a, p.b);
        if (fabs(d) > eps) {
            pt res;
            res.x = -det(c, b, p.c, p.b) / d;
            res.y = -det(a, c, p.a, p.c) / d;
            return res;
        }
        assert(false);
    }

    bool equivalent(const line &p) const {
        return fabs(det(a, b, p.a, p.b)) < eps
               && fabs(det(a, c, p.a, p.c)) < eps
               && fabs(det(b, c, p.b, p.c)) < eps;
    }

    bool parallel(const line& p) const {
        return fabs(det (a, b, p.a, p.b)) < eps;
    }
};

pair<ld, ld> intersect(pair<ld, ld> a, pair<ld, ld> b) {
    return {max(min(a.first, a.second), min(b.first, b.second)),
            min(max(a.first, a.second), max(b.first, b.second))};
}

bool check(vector<line>& a) {
    random_shuffle(a.begin(), a.end());
    a.push_back(line({-inf, -inf}, {-inf, inf}));
    a.push_back(line({-inf, inf}, {inf, inf}));
    a.push_back(line({inf, inf}, {inf, -inf}));
    a.push_back(line({inf, -inf}, {-inf, -inf}));
    reverse(a.begin(), a.end());
    for (int i = 0; i < 4; ++i) {
        a[i].a *= -1;
        a[i].b *= -1;
        a[i].c *= -1;
    }
    pt p(-inf, inf);
    int res = 0;
    for (int i = 4; i < a.size(); ++i) {
        if (a[i].get_val(p) < eps) {
            pair<ld, ld> xx = {-inf, inf};
            for (int j = 0; j < i; ++j) {
                if (a[i].parallel(a[j])) {
                    pt q(1.0, a[i].get_y(1.0));
                    if (a[j].get_val(q) < eps)
                        return false;
                } else {
                    pt q = a[i].intersect(a[j]);
                    pt q2(q.x - 1.0, a[i].get_y(q.x - 1.0));
                    if (a[j].get_val(q2) > -eps) {
                        xx = intersect(xx, {-inf, q.x});
                    } else {
                        xx = intersect(xx, {q.x, inf});
                    }
                }
                if (xx.second - xx.first < -eps)
                    return false;
            }
            if (xx.second - xx.first > -eps) {
                pt p1(xx.first, a[i].get_y(xx.first));
                pt p2(xx.second, a[i].get_y(xx.second));
                if (p1.y - p2.y > -eps)
                    p = p1;
                else
                    p = p2;
            } else {
                return false;
            }
        }
        ++res;
    }
    return true;
}
\end{lstlisting}
\subsection{string}
\subsubsection{duval}
\begin{lstlisting}
/*
 * This file contains realization of Duval algorithm.
 * For given string 's' finds minimal cyclic shift in O(n).
 *
 * Requirements: "string"
 */

string min_cyclic_shift(string s) {
    s += s;
    int n = int(s.length());
    int i = 0;
    int ans = 0;
    while (i < n / 2) {
        ans = i;
        int j = i + 1;
        int k = i;
        while (j < n && s[k] <= s[j]) {
            if (s[k] < s[j])
                k = i;
            else
                ++k;
            ++j;
        }
        while (i <= k) i += j - k;
    }
    return s.substr(ans, n / 2);
}
\end{lstlisting}
\subsubsection{hashing}
\begin{lstlisting}
/*
 *
 * Polynomial hashing.
 *
 * Complexity: O(N).
 *
 */


const ll mod = (ll)1e9 + 7;

ll p = rand() % 100 + 100;


ll h[MAXN], deg[MAXN];

void hashing(string s) {

    h[0] = 0, deg[0] = 1;
    for (int i = 0; i < s.size(); i++) {
        h[i + 1] = (h[i] * p) % mod + s[i];
        h[i + 1] %= mod;
        deg[i + 1] = (deg[i] * p) % mod;
    }
    auto get_hash = [&]( int l, int r) { // [l..r]
        return ((h[r + 1] - (h[l] * deg[r - l + 1]) % mod) % mod + mod) % mod;
    };
}
\end{lstlisting}
\subsubsection{suffix auto}
\begin{lstlisting}
/*
 *  This file contains Suffix automation.
 *
 *  Requirements: "string", "map"
 */

struct State{
    int len;
    int link;
    int endpos;
    int cnt = 0;
    map<char, int> next;
};

State st[2 * N - 1];

string s;

bool cmp(int a, int b)  {
    return st[a].len > st[b].len;
}

void build(){
    int n = s.size();
    st[0].link = -1;
    st[0].endpos = -1;
    int last = 0;
    int size = 1;
    for(int i = 0; i < n; i++){
        char c = s[i];
        int cur = size++;
        st[cur].cnt = 1;
        st[cur].len = st[last].len + 1;
        int p;
        for (p=last; p!=-1 && !st[p].next.count(c); p=st[p].link)
            st[p].next[c] = cur;
        if (p == -1)
            st[cur].link = 0;
        else {
            int q = (st[p].next.count(c) > 0 ? st[p].next[c] : -1);
            if (st[p].len + 1 == st[q].len)
                st[cur].link = q;
            else {
                int clone = size++;
                st[clone].len = st[p].len + 1;
                st[clone].next = st[q].next;
                st[clone].link = st[q].link;
                for (; p!=-1 && (st[p].next.count(c) ? st[p].next[c] == q : false); p=st[p].link)
                    st[p].next[c] = clone;
                st[q].link = st[cur].link = clone;
            }
        }
        last = cur;
    }
    vector<int> ord(size);
    for (int i = 0; i < size; ++i) {
        ord[i] = i;
    }
    sort(ord.begin(), ord.end(), cmp);
    for (auto i : ord) {
        if (st[i].link > 0)
            st[st[i].link].cnt += st[i].cnt;
    }
}

\end{lstlisting}
\subsubsection{manacher}
\begin{lstlisting}
/*
 *  This file contains Manacher algo realization.
 *  For given string 's' finds all subpalindromes in O(n).
 *
 *  Requirements: "string", "vector"
 */

void manacher(string s) {
    int n = int(s.size());
    vector<int> d1(n);
    vector<int> d2(n);
    int l = -1;
    int r = -1;
    for(int i = 0; i < n; i++) {
        int k = i > r ? 1 : min(d1[l + r - i], r - i);
        while (0 <= i - k && i + k < n && s[i - k] == s[i + k])
            k++;
        d1[i] = k;
        if(i + k - 1 > r) {
            r = i + k - 1;
            l = i - k + 1;
        }
    }
    for(int i = 0; i < n; i++) {
        int k = i > r? 0 : min(d2[l + r - i + 1], r - i + 1);
        while (0 <= i - k - 1 && i + k < n && s[i + k] == s[i - k - 1])
            k++;
        d2[i] = k;
        if(i + k - 1 > r) {
            l = i - k;
            r = i + k - 1;
        }
    }
}
\end{lstlisting}
\subsubsection{prefix function}
\begin{lstlisting}
 /*
  *
  * Calculate prefix function of string.
  *
  * Complexity: O(N).
  *
  *
  */

vector<int> prefix_function(string s) {
	int n = (int)s.length();
	vector<int> pi(n);
	for (int i = 1; i < n; ++i) {
		int j = pi[i - 1];
		while (j > 0 && s[i] != s[j])
			j = pi[j - 1];
		if (s[i] == s[j])
            ++j;
		pi[i] = j;
	}
	return pi;
}
\end{lstlisting}
\subsection{data structures}
\subsubsection{cartesian tree push}
\begin{lstlisting}
/*
 * Cartesian tree with implicit keys and lazy propagation.
 * Operations: insert, add on segment, sum on segment, left-right traversal.
 *
 * Build complexity: O(NlogN)
 * queries complexity: O(logN)
 *
 */

#define ll long long

struct Node {
    ll push, sum;
    ll val;
    ll size, y;
    Node *left, *right;
    Node(ll val = 0)
            : val(val), size(1), y(rand()),
              push(-1), sum(val),
              left(nullptr), right(nullptr) { }
};


ll get_push(Node *node) {
    return node != nullptr? node->push : -1;
}

ll get_size(Node *node) {
    return node != nullptr? node->size : 0;
}

ll get_sum(Node *node) {
    return node != nullptr? node->sum : 0;
}

void push(Node *v) {
    if (v == nullptr || v->push == -1)
        return;
    if (v->left)
        v->left->push = v->push;
    if (v->right)
        v->right->push = v->push;
    v->val = v->push;
    v->sum = v->push * v->size;
    v->push = -1;
}

void recalc(Node *node) {
    node->size = get_size(node->left) + get_size(node->right) + 1;
    if (node->push != -1)
        node->sum = node->push * node->size;
    else {
        ll left_sum, right_sum;
        if (get_push(node->left) != -1)
            left_sum = get_push(node->left) * get_size(node->left);
        else left_sum = get_sum(node->left);
        if (get_push(node->right) != -1)
            right_sum = get_push(node->right) * get_size(node->right);
        else right_sum = get_sum(node->right);
        node->sum = left_sum + right_sum + node->val;
    }
}

pair<Node*, Node*> split(Node *root, ll val) {
    push(root);
    if (root == nullptr)
        return mp(nullptr, nullptr);
    ll x = get_size(root->left);
    if (x <= val) {
        pair<Node*, Node*> splitted = split(root->right, val - x - 1);
        root->right = splitted.first;
        recalc(root);
        return mp(root, splitted.second);
    } else {
        pair<Node*, Node*> splitted = split(root->left, val);
        root->left = splitted.second;
        recalc(root);
        return mp(splitted.first, root);
    }
}

Node* merge(Node *left, Node *right) {
    push(left);
    push(right);
    if (left == nullptr)
        return right;
    if (right == nullptr)
        return left;
    if (left->y > right->y) {
        left->right = merge(left->right, right);
        recalc(left);
        return left;
    } else {
        right->left = merge(left, right->left);
        recalc(right);
        return right;
    }
}

void insert(Node *&root, ll pos, ll val) {
    pair<Node*, Node*> splitted = split(root, pos - 1);
    root = merge(merge(splitted.first, new Node(val)), splitted.second);
}

void update(Node *&root, ll l, ll r, ll add) {
    pair<Node*, Node*> splitted = split(root, l - 1);
    pair<Node*, Node*> splitted2 = split(splitted.second, r - l);
    splitted2.first->push = add;
    root = merge(splitted.first, merge(splitted2.first, splitted2.second));
}

ll get(Node *&root, ll l, ll r) {
    pair<Node*, Node*> splitted = split(root, l - 1);
    pair<Node*, Node*> splitted2 = split(splitted.second, r - l);
    ll res = splitted2.first->sum;
    root = merge(splitted.first, merge(splitted2.first, splitted2.second));
    return res;
}

vector<ll> all;

void get_all(Node *root) {
    if (root == nullptr)
        return;
    get_all(root->left);
    all.push_back(root->val);
    get_all(root->right);
}

\end{lstlisting}
\subsubsection{persistent segment tree}
\begin{lstlisting}
/*
 *
 * Persistent segment tree.
 *
 * Complexity: O(logN).
 *
 * Usage:
 *   - Init: Create array of roots and place empty Node()
 *   - Update: roots.push_back(update(roots.back(), ...))
 *   - Get: get(roots[r], ...) - get(roots[l - 1], ...)
 *
 */


struct Node {
    int val, sum;
    Node *left, *right;
    Node(int val = 0) : val(val), sum(0), left(nullptr), right(nullptr) {}
    Node(const Node& v) : val(v.val), sum(v.sum), left(v.left), right(v.right) {}
};

int get_sum(Node *node) {
    return node? node->sum : 0;
}

Node* update(Node *v, int tl, int tr, int pos, int val) {
    Node *u = v? new Node(*v) : new Node();
    if (tl == tr) {
        u->sum = val;
        u->val = val;
        return u;
    }
    int tm = (tl + tr) / 2;
    if (pos <= tm)
        u->left = update(u->left, tl, tm, pos, val);
    else
        u->right = update(u->right, tm + 1, tr, pos, val);
    u->sum = get_sum(u->left) + get_sum(u->right);
    return u;
}


int get(Node *v, int tl, int tr, int l, int r) {
    if (v == nullptr)
        return 0;
    if (l > r)
        return 0;
    if (tl == l && tr == r)
        return v->sum;
    int tm = (tl + tr) / 2;
    return get(v->left, tl, tm, l, min(r, tm)) +
           get(v->right, tm + 1, tr, max(l, tm + 1), r);
}

\end{lstlisting}
\subsubsection{segment tree}
\begin{lstlisting}
/*
 * Segment tree with lazy propagation.
 *
 * Build complexity: O(N)
 * queries complexity: O(logN)
 *
 */


template<class ValueType>
class SumSegmentTree {
 public:
     SumSegmentTree() = default;
     SumSegmentTree(int sz) : tree(sz * 4), pushing(sz * 4), size(sz) {}
     SumSegmentTree(vector<ValueType> a, int size) : SumSegmentTree(size) {
         build(a, 1, 0, size);
     }

     int add(int l, int r, ValueType delta) {
         return add(1, 0, size - 1, l, r, delta);
     }

     int add(int p, ValueType delta) {
         return add(1, 0, size - 1, p, p, delta);
     }

     int get(int l, int r) {
         return get(1, 0, size - 1, l, r);
     }

 private:
     ValueType merge(int v, int u) {
         return tree[v] + pushing[v] + tree[u] + pushing[u];
     }

     void push(int v) {
         tree[v] += pushing[v];
         pushing[v + v] += pushing[v];
         pushing[v + v + 1] += pushing[v];
         pushing[v] = 0;
     }

     void build(vector<ValueType>& a, int v, int tl, int tr) {
         if (tl == tr) {
             tree[v] = a[tl];
         } else {
             int mid = (tl + tr) / 2;
             build(a, v + v, tl, mid);
             build(a, v + v + 1, mid + 1, tr);
             tree[v] = merge(v + v, v + v + 1);
         }
     }

     void add(int v, int tl, int tr, int l, int r, ValueType delta) {
         if (l > r) {
             return;
         }
         if (l == tl && r == tr) {
             pushing[v] += delta;
             return;
         }
         push(v);
         int mid = (tl + tr) / 2;
         add(v + v, tl, mid, l, min(r, mid), delta);
         add(v + v + 1, mid + 1, tr, max(l, mid + 1), r, delta);
         tree[v] = merge(v + v, v + v + 1);
     }

     ValueType get(int v, int tl, int tr, int l, int r) {
         if (l > r) {
             return 0;
         }
         if (l == tl && r == tr) {
             return tree[v] + pushing[v];
         }
         push(v);
         int mid = (tl + tr) / 2;
         return get(v + v, tl, mid, l, min(r, mid)) +
                 get(v + v + 1, mid + 1, tr, max(l, mid + 1), r);
     }

     vector<ValueType> tree;
     vector<ValueType> pushing;
     size_t size;
};

\end{lstlisting}
\subsubsection{cube}
\begin{lstlisting}
/*
 * Cube class
 * state[3] - down side
 * state[1] - up side
 * state[4] - left side
 * state[5] - right side
 * state[2] - frost side
 * state[0] - back side
 *
 *
 */


int dx[4] = {-1, 0, 0, 1};
int dy[4] = {0, -1, 1, 0};
int updown[4] = {0, 1, 2, 3};
int leftright[4] = {1, 4, 3, 5};
int now[4];

struct Cube {
    int state[6];

    Cube() {
        for (int i = 0; i < 6; ++i) {
            state[i] = i;
        }
    }

    void rotate(int dir) {
        if (dir == 1) {
            for (int i = 0; i < 4; ++i) {
                now[i] = state[updown[i]];
            }
            reverse(now, now + 3);
            reverse(now, now + 4);
            for (int i = 0; i < 4; ++i) {
                state[updown[i]] = now[i];
            }
        }
        if (dir == 2) {
            for (int i = 0; i < 4; ++i) {
                now[i] = state[updown[i]];
            }
            reverse(now + 1, now + 4);
            reverse(now, now + 4);
            for (int i = 0; i < 4; ++i) {
                state[updown[i]] = now[i];
            }
        }
        if (dir == 3) {
            for (int i = 0; i < 4; ++i) {
                now[i] = state[leftright[i]];
            }
            reverse(now + 1, now + 4);
            reverse(now, now + 4);
            for (int i = 0; i < 4; ++i) {
                state[leftright[i]] = now[i];
            }
        }
        if (dir == 0) {
            for (int i = 0; i < 4; ++i) {
                now[i] = state[leftright[i]];
            }
            reverse(now, now + 3);
            reverse(now, now + 4);
            for (int i = 0; i < 4; ++i) {
                state[leftright[i]] = now[i];
            }
        }
    }
    
    vector<int> get_state() const {
        vector<int> s;
        for (int i = 0; i < 6; ++i) {
            s.push_back(state[i]);
        }
        return s;
    }

    bool operator<(const Cube& other) const {
        return get_state() < other.get_state();
    }
};

\end{lstlisting}
\subsubsection{cartesian tree}
\begin{lstlisting}
/*
 * Cartesian tree with implicit keys.
 * Operation: min_element on segment.
 *
 * Build complexity: O(NlogN)
 * split/merge complexity: O(logN)
 *
 * Usage:
 *   - initialize root with nullptr
 *   - split(root, x) returns tree with keys <= x
 *
 */


struct Node {
    int val;
    int size, y;
    int min_e, min_i;
    Node *left, *right;
    Node(int val = 0)
            : val(val), size(1), y(rand()),
              min_e(val), min_i(0),
              left(nullptr), right(nullptr) { }
};

int get_size(Node *node) {
    return node != nullptr? node->size : 0;
}

void recalc(Node *node) {
    node->size = get_size(node->left) + get_size(node->right) + 1;
    int mv = node->val;
    int pos = get_size(node->left);
    if (node->left != nullptr) {
        if (mv >= node->left->min_e) {
            mv = node->left->min_e;
            pos = node->left->min_i;
        }
    }
    if (node->right != nullptr) {
        if (mv > node->right->min_e) {
            mv = node->right->min_e;
            pos = get_size(node->left) + 1 + node->right->min_i;
        }
    }
    node->min_e = mv;
    node->min_i = pos;
}

pair<Node*, Node*> split(Node *root, int val) {
    if (root == nullptr)
        return mp(nullptr, nullptr);
    int x = get_size(root->left);
    if (x <= val) {
        pair<Node*, Node*> splitted = split(root->right, val - x - 1);
        root->right = splitted.first;
        recalc(root);
        return mp(root, splitted.second);
    } else {
        pair<Node*, Node*> splitted = split(root->left, val);
        root->left = splitted.second;
        recalc(root);
        return mp(splitted.first, root);
    }
}

Node* merge(Node *left, Node *right) {
    if (left == nullptr)
        return right;
    if (right == nullptr)
        return left;
    if (left->y > right->y) {
        left->right = merge(left->right, right);
        recalc(left);
        return left;
    } else {
        right->left = merge(left, right->left);
        recalc(right);
        return right;
    }
}

\end{lstlisting}
\subsubsection{mo}
\begin{lstlisting}
/*
 * This file contains Mo's algorithm signature.
 * This is sqrt decomposition of queries.
 *
 * Complexity: sorting O(nlogn), suppose add/remove comlexity is O(g), then
 *             overall comlexity is O(Q \sqrt{Q} g)
 *
 * Requirements: "algorithm"
 */


sort(q, q + Q, cmp);

int l = 0;
int r = 0;
for(int i = 0; i < Q; i++){
    int L = q[i].l;
    int R = q[i].r;
    while(curl < L){
        // remove 'l'
        ++l;
    }
    while(curl > L){
        // add 'l - 1'
        --l;
    }
    while(curr <= R){
        // add 'r'
        r++;
    }
    while(curr > R + 1){
        // remove 'r - 1'
        curr--;
    }
}

\end{lstlisting}
\subsection{graph}
\subsubsection{find scc}
\begin{lstlisting}
/*
 *
 * Find strongly connected components.
 *
 * Complexity: O(N + M).
 *
 * Usage:
 *   - g is adjacency table of graph, gr is g reversed
 *
 */



vector<vector<int>> g, gr;
vector<char> used;
vector<int> order, component;
 
void dfs1 (int v) {
	used[v] = true;
	for (int i = 0; i < g[v].size(); ++i)
		if (!used[g[v][i]])
			dfs1(g[v][i]);
	order.push_back (v);
}
 
void dfs2 (int v) {
	used[v] = true;
	component.push_back (v);
	for (int i = 0; i < gr[v].size(); ++i)
		if (!used[gr[v][i]])
			dfs2(gr[v][i]);
}
 
void find_scc() {
	for (int i = 0; i < n; ++i)
		if (!used[i])
			dfs1(i);
	used.assign (n, false);
	for (int i = 0; i < n; ++i) {
		int v = order[n - 1 - i];
		if (!used[v]) {
			dfs2 (v);
            // print component
			component.clear();
		}
	}
}
\end{lstlisting}
\subsubsection{dinic}
\begin{lstlisting}
/*
 * This file contains realization of Dinic algorithm.
 * For given graph 'g' finds maximal flow in O(EV^2).
 *
 * MAXN - maximal number of vertexes in G
 * s - start vertex
 * t - end vertex
 *
 * dinic() - runs algorithm, returns max flow
 * add_edge(a, b, cap) - add edge from 'a' to 'b' with capacity 'cap'
 *
 * Requirements: "vector", "cstring"
 */

const long long MAXN = 505;
const long long inf = 1e18;

struct edge {
    long long a, b, cap, flow;
};

long long n, s, t, d[MAXN], ptr[MAXN], q[MAXN];
vector<edge> e;
vector<long long> g[MAXN];

void add_edge(long long a, long long b, long long cap) {
    edge e1 = { a, b, cap, 0 };
    edge e2 = { b, a, 0, 0 };
    g[a].push_back(e.size());
    e.push_back(e1);
    g[b].push_back(e.size());
    e.push_back(e2);
}

bool bfs() {
    long long qh = 0, qt = 0;
    q[qt++] = s;
    memset(d, -1, MAXN * sizeof d[0]);
    d[s] = 0;
    while (qh < qt && d[t] == -1) {
        long long v = q[qh++];
        for (size_t i = 0; i < g[v].size(); ++i) {
            long long id = g[v][i],
                    to = e[id].b;
            if (d[to] == -1 && e[id].flow < e[id].cap) {
                q[qt++] = to;
                d[to] = d[v] + 1;
            }
        }
    }
    return d[t] != -1;
}

long long dfs(long long v, long long flow) {
    if (!flow) return 0;
    if (v == t) return flow;
    for (; ptr[v] < g[v].size(); ++ptr[v]) {
        long long id = g[v][ptr[v]],
                to = e[id].b;
        if (d[to] != d[v] + 1)  continue;
        long long pushed = dfs(to, min(flow, e[id].cap - e[id].flow));
        if (pushed) {
            e[id].flow += pushed;
            e[id ^ 1].flow -= pushed;
            return pushed;
        }
    }
    return 0;
}

long long dinic() {
    long long flow = 0;
    for (long long lim = 1 << 30; lim >= 1;) {
        if (!bfs()) {
            lim >>= 1;
            continue;
        }
        memset(ptr, 0, MAXN * sizeof ptr[0]);
        while (long long pushed = dfs(s, lim))
            flow += pushed;
    }
    return flow;
}

\end{lstlisting}
\subsubsection{min cost max flow}
\begin{lstlisting}
/*
 *
 * Find min cost max flow
 *
 * Complexity: ~ O(MN^3).
 */


const int N = (int)250;

struct Edge {
    int to;
    int cap;
    int cost;
    int flow;
    int id;
    int rid;
    int from;
    Edge() = default;
    Edge(int to, int cap, int cost, int flow, int id, int rid, int from) :
                to(to), cap(cap),
                cost(cost), flow(flow),
                id(id), rid(rid), from(from) {}
};

int s = N - 4, t = N - 3;

vector<int> g[N];

vector<Edge> e;

void add_edge(int u, int v, int cap, int cost) {
    g[u].push_back(e.size());
    e.push_back(Edge(v, cap, cost, 0, e.size(), g[v].size(), u));
    g[v].push_back(e.size());
    e.push_back(Edge(u, 0, -cost, 0, e.size(), g[u].size() - 1, v));
}

pair<int, int> find_flow() {
    int flow = 0;
    int cost = 0;
    while (flow < sum) {
        vector<int> id(N, 0);
        vector<int> d(N, inf);
        vector<int> q(N);
        vector<int> p(N);
        vector<size_t> p_edge(N);
        int qh = 0, qt = 0;
        q[qt++] = s;
        d[s] = 0;
        while (qh != qt) {
            int v = q[qh++];
            id[v] = 2;
            if (qh == N) qh = 0;
            for (size_t i = 0; i < g[v].size(); ++i) {
                Edge& r = e[g[v][i]];
                if (r.flow < r.cap && d[v] + r.cost < d[r.to]) {
                    d[r.to] = d[v] + r.cost;
                    if (id[r.to] == 0) {
                        q[qt++] = r.to;
                        if (qt == N) qt = 0;
                    } else if (id[r.to] == 2) {
                        if (--qh == -1) qh = N - 1;
                        q[qh] = r.to;
                    }
                    id[r.to] = 1;
                    p[r.to] = v;
                    p_edge[r.to] = i;
                }
            }
        }
        if (d[t] == inf)  break;
        int addflow = sum - flow;
        for (int v = t; v != s; v = p[v]) {
            int pv = p[v];
            size_t pr = p_edge[v];
            addflow = min(addflow, e[g[pv][pr]].cap - e[g[pv][pr]].flow);
        }
        int add_cost = 0;
        for (int v = t; v != s; v = p[v]) {
            int pv = p[v];
            size_t pr = p_edge[v];
            size_t r = e[g[pv][pr]].rid;
            e[g[pv][pr]].flow += addflow;
            e[g[v][r]].flow -= addflow;
            add_cost += e[g[pv][pr]].cost * addflow;
        }
        flow += addflow;
        cost += add_cost;
    }
    return {flow, cost};
}

\end{lstlisting}
\subsubsection{centroid}
\begin{lstlisting}
/*
 * Centroid decomposition.
 *
 * build Complexity: O(N + M).
 *
 */



const int N = 3e4 + 1;

vector<int> g[N];
int sz[N];
bool used[N];
int depth[N];
set<pair<int, int>> in_centroid[N];
vector<pair<int, int>> centroids[N];

void dfs_sz(int v, int prev) {
    sz[v] = 1;
    for (auto to : g[v]) {
        if (used[to] || to == prev)
            continue;
        dfs_sz(to, v);
        sz[v] += sz[to];
    }
}

int find_centroid(int v, int prev, int size) {
    for (auto to : g[v]) {
        if (used[to] || to == prev)
            continue;
        if (sz[to] > size / 2) {
            return find_centroid(to, v, size);
        }
    }
    return v;
}

void relax(int v, int centroid, int prev, int dist) {
    for (auto to : g[v]) {
        if (used[to] || to == prev)
            continue;
        relax(to, centroid, v, dist + 1);
    }
    in_centroid[centroid].insert(make_pair(dist, v));
    centroids[v].push_back(make_pair(dist, centroid));
}

void divide(int v) {
    dfs_sz(v, -1);
    v = find_centroid(v, -1, sz[v]);
    used[v] = true;
    relax(v, v, -1, 0);
    for (auto to : g[v]) {
        if (used[to])
            continue;
        divide(to);
    }
}
\end{lstlisting}
\subsubsection{bridges}
\begin{lstlisting}
/*
 * Find bridges.
 *
 * Complexity: O(N + M).
 *
 */


vector<vector<int>> g;
vector<char> used;
vector<int> timer, tin, fup;
 
void dfs (int v, int p = -1) {
	used[v] = true;
	tin[v] = fup[v] = timer++;
	for (int i = 0; i < g[v].size(); ++i) {
		int to = g[v][i];
		if (to == p)  continue;
		if (used[to])
			fup[v] = min(fup[v], tin[to]);
		else {
			dfs (to, v);
			fup[v] = min(fup[v], fup[to]);
			if (fup[to] > tin[v])
				// edge (v - to) is bridge
		}
	}
}
 
void find_bridges() {
	timer = 0;
	for (int i = 0; i < n; ++i)
		used[i] = false;
	for (int i = 0; i < n; ++i)
		if (!used[i])
			dfs(i);
}
\end{lstlisting}
\subsubsection{kuhn}
\begin{lstlisting}
/*
 * This file contains realiztaion of Kuhn algorithm.
 * For given bipartite graph 'g' finds maximum matching.
 * 
 * This implementation expects a bipartite graph comprised of disjount sets G1 and G2 with sizes n and k.
 * g[i] contains all possible matches (from G2) for a node i (from G1).
 * 
 * NOTE: This algorithm is faster if n < k.
 *
 * Comlexity: O(EV)
 *
 * Requirements: "vector"
 */

int n, k;
vector<vector<int>> g;
vector<int> mt;
vector<char> used;

bool kuhn_step(int v) {
    if (used[v]) return false;
    used[v] = true;
    for (auto to : g[v]) {
        if (mt[to] == -1 || kuhn_step(mt[to])) {
            mt[to] = v;
            return true;
        }
    }
    return false;
}

void kuhn() {
    mt.assign(k, -1);
    for (int v = 0; v < n; ++v) {
        used.assign(n, false);
        kuhn_step(v);
    }
}

\end{lstlisting}
\subsubsection{hungarian}
\begin{lstlisting}
/*
 * This file contains realization of hungarian algorithm.
 * 
 * Complexity: O(N^3)
 *
 * Requirements: "vector"
 */


int hungarian(vector<vector<int>> g, int n) {
    vector<int> u(n + 1), v(n + 1), p(n + 1), way(n + 1);
    for (int i = 1; i <= n; ++i) {
        p[0] = i;
        int j0 = 0;
        vector<int> minv(n + 1, inf);
        vector<char> used(n + 1, false);
        do {
            used[j0] = true;
            int i0 = p[j0], delta = inf, j1;
            for (int j = 1; j <= n; ++j)
                if (!used[j]) {
                    int cur = g[i0][j] - u[i0] - v[j];
                    if (cur < minv[j])
                        minv[j] = cur,  way[j] = j0;
                    if (minv[j] < delta)
                        delta = minv[j], j1 = j;
                }
            for (int j = 0; j <= n; ++j)
                if (used[j])
                    u[p[j]] += delta, v[j] -= delta;
                else
                    minv[j] -= delta;
            j0 = j1;
        } while (p[j0] != 0);
        do {
            int j1 = way[j0];
            p[j0] = p[j1];
            j0 = j1;
        } while (j0);
    }
    return -v[0];
}

\end{lstlisting}
